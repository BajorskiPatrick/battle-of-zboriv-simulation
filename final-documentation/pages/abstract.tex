% You need to change the text for abstract
Edycja tego rozdziału w pliku: pages/abstract.tex\\

Abstrakt jest zamieszczony w charakterze informacyjnym. Można go zakomentować w pliku: ju\_cse\_report.sty. Odpowiednie miejsce jest zaznaczone komentarzem. Oczywiście jeśli zespół chce może tu wstawić własny abstrakt.

Kod źródłowy Latex dla raportu jest zorganizowany w spośób modułowy. Bazą dla tego szablonu jest wersja oryginalna dostepna pod adresem \url{https://www.overleaf.com/latex/templates/ju-cse-report-template/ctzvgmyrmbzq}.

\textbf{Przeczytaj poniższe instrukcje, aby zrozumieć strukturę dokumentu.}

\texttt{chapters}   --: Zawiera rozdziały raportu.

\texttt{figures}    --: Folder na rysunki i ilustracje.

\texttt{pages}      --: Pliki definuijące zawartośc dla streszczenia, bibliografii, itp. 

\texttt{parameters}  --: W tym folderze można dodać informacje o pracy dyplomowej, takie jak imię i nazwisko autora, ID,
stopień naukowy, sesja, nazwisko opiekuna. 

\texttt{ju\_cse\_report.sty}  --: Można definiować ustawienia stylu dokumentu. 

\texttt{ju\_cse\_report\_Ref.bib}  --: Lista referencji. Bibliografia w formacie Bib\TeX. 

\texttt{report.tex}      --: Makro struktura raportu. Jesli grupa koniecznie potrzebuje można tu dodać dodatkowe elementy np. rozdziały.

\endinput
