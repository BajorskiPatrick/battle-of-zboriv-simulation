\chapter{PODSUMOWANIE}\label{ch:conclusion}

W ramach projektu udało się zrealizować kompletną, działającą symulację bitwy historycznej. Osiągnięto wszystkie założone cele, a przeprowadzone eksperymenty dostarczyły istotnych danych analitycznych:

\begin{enumerate}
    \item \textbf{Wiarygodny model jednostek:} Zaimplementowano 13 typów jednostek o unikalnych parametrach. Eksperymenty syntetyczne potwierdziły poprawność balansu, wykazując m.in. dominację elitarnej Husarii nad liczniejszą Czernią (95\% zwycięstw) oraz wysoką efektywność Dragonii w starciu z lekką jazdą.
    \item \textbf{Wpływ środowiska:} Potwierdzono hipotezę o kluczowym wpływie pogody na losy bitwy. Symulacje wykazały, że w warunkach idealnych Armia Koronna wygrywa w 100\% przypadków, podczas gdy gęsta mgła odwraca te proporcje, dając 80\% zwycięstw stronie kozackiej poprzez neutralizację przewagi ogniowej.
    \item \textbf{Narzędzia analityczne:} Stworzono system zbierania danych i wizualizacji (dashboard, heatmap), który pozwolił na ilościową ocenę starć, w tym analizę wpływu "mgły wojny" i terenu na efektywność poszczególnych formacji.
\end{enumerate}

Projekt pokazał, że narzędzia symulacji systemów dyskretnych (Mesa) mogą być z powodzeniem stosowane w cyfrowej humanistyce do weryfikacji hipotez historycznych. Wyniki eksperymentów "Jakość vs Ilość" oraz "Siła Ognia" dostarczyły ciekawych obserwacji na temat taktyki XVII-wiecznej, wskazując na granice efektywności dyscypliny w starciu z przewagą liczebną.

Jako kierunek dalszego rozwoju wskazać można implementację bardziej zaawansowanych algorytmów grupowych (boids) dla lepszego odwzorowania ruchu formacji oraz wprowadzenie trybu multiplayer, w którym użytkownicy mogliby wydawać rozkazy w czasie rzeczywistym, zmieniając cele strategiczne agentów.
