\chapter{PODSUMOWANIE}\label{ch:conclusion}

W ramach projektu udało się zrealizować kompletną, działającą symulację bitwy historycznej. Osiągnięto wszystkie założone cele:
\begin{enumerate}
    \item Stworzono wiarygodny model jednostek historycznych.
    \item Zaimplementowano wpływ środowiska na przebieg walki (deszcz, teren).
    \item Dostarczono narzędzia analityczne potwierdzające tezę, że warunki pogodowe mogły odwrócić losy bitwy pod Zborowem.
\end{enumerate}

Projekt pokazał, że narzędzia symulacji systemów dyskretnych (Mesa) mogą być z powodzeniem stosowane w cyfrowej humanistyce i analizie historycznej. 

Jako kierunek dalszego rozwoju wskazać można implementację algorytmów grupowych (boids) dla lepszego odwzorowania ruchu formacji oraz wprowadzenie trybu multiplayer, w którym użytkownicy mogliby wydawać rozkazy w czasie rzeczywistym, zmieniając cele strategiczne agentów.