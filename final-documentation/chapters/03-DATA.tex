\chapter{DANE PORÓWNAWCZE I WEJŚCIOWE} \label{ch:data}

\section{Dane wejściowe symulacji}

Danymi wejściowymi dla modelu są parametryzowane charakterystyki jednostek oraz cyfrowa reprezentacja terenu. Dane te zostały pozyskane z literatury historycznej i przetworzone na formaty cyfrowe (pliki JSON, TMX).

\subsection{Pełna charakterystyka jednostek}
W symulacji zaimplementowano 13 unikalnych typów jednostek, podzielonych na dwie frakcje: Armię Koronną oraz Wojska Kozacko-Tatarskie. Parametry te zostały zaszyte w pliku konfiguracyjnym modelu i determinują zachowanie agentów. Poniższe tabele prezentują kompletny zestaw danych.

\begin{table}[!ht]
    \caption{\label{tab:units_attributes} Podstawowe atrybuty jednostek (HP, Morale, Dyscyplina, Obrona, Szybkość).}
    \centering
    \begin{tabular}{l c c c c c}
     \toprule
    \textbf{Jednostka} & \textbf{HP} & \textbf{Morale} & \textbf{Dysc.} & \textbf{Obrona} & \textbf{Szybkość} \\
    \cmidrule[0.4pt](r{0.125em}){1-1}%
    \cmidrule[0.4pt](lr{0.125em}){2-6}%
    \textbf{Armia Koronna} & & & & & \\
    Husaria & 150 & 140 & 95 & 8 & 6 \\
    Pancerni & 120 & 110 & 85 & 5 & 7 \\
    Rajtaria & 110 & 100 & 90 & 6 & 6 \\
    Dragonia & 100 & 95 & 85 & 4 & 5 \\
    Piechota Niemiecka & 110 & 100 & 95 & 6 & 3 \\
    Pospolite Ruszenie & 90 & 50 & 20 & 2 & 6 \\
    Czeladź Obozowa & 60 & 90 & 40 & 0 & 5 \\
    Artyleria Koronna & 50 & 90 & 90 & 0 & 1 \\
    \hline
    \textbf{Kozacy/Tatarzy} & & & & & \\
    Jazda Tatarska & 85 & 80 & 70 & 1 & 9 \\
    Piechota Kozacka & 115 & 110 & 80 & 3 & 4 \\
    Jazda Kozacka & 100 & 90 & 75 & 3 & 7 \\
    Czerń & 70 & 60 & 40 & 0 & 5 \\
    Artyleria Kozacka & 40 & 80 & 80 & 0 & 1 \\
    \bottomrule
    \end{tabular}
\end{table}

\begin{table}[!ht]
    \caption{\label{tab:units_combat} Zdolności bojowe jednostek (Atak, Zasięg, Amunicja, Szybkostrzelność).}
    \centering
    \resizebox{\textwidth}{!}{
    \begin{tabular}{l c c c c c l}
     \toprule
    \textbf{Jednostka} & \textbf{Atak Wręcz} & \textbf{Atak Dyst.} & \textbf{Zasięg} & \textbf{Ammo} & \textbf{RoF} & \textbf{Uwagi} \\
    \cmidrule[0.4pt](r{0.125em}){1-1}%
    \cmidrule[0.4pt](lr{0.125em}){2-7}%
    \textbf{Armia Koronna} & & & & & & \\
    Husaria & 100 & 0 & 1 & 0 & 1.0 & Elitarna ciężka jazda przełamująca \\
    Pancerni & 70 & 0 & 1 & 0 & 1.0 & Jazda średniozbrojna, uniwersalna \\
    Rajtaria & 40 & 30 & 3 & 12 & 0.8 & Ciężka jazda z bronią palną \\
    Dragonia & 30 & 25 & 4 & 15 & 1.2 & Mobilna piechota konna \\
    Piechota Niemiecka & 25 & 35 & 5 & 20 & 1.3 & Wysoka dyscyplina, silny ogień \\
    Pospolite Ruszenie & 20 & 10 & 2 & 5 & 0.8 & Niska dyscyplina, podatność na panikę \\
    Czeladź Obozowa & 25 & 0 & 1 & 0 & 1.0 & Słabo uzbrojona, zdeterminowana \\
    Artyleria Koronna & 5 & 150 & 15 & 30 & 0.25 & Potężna siła ognia, bardzo wolna \\
    \hline
    \textbf{Kozacy/Tatarzy} & & & & & & \\
    Jazda Tatarska & 30 & 15 & 4 & 40 & 1.8 & Szybcy łucznicy \\
    Piechota Kozacka & 35 & 35 & 5 & 25 & 1.4 & Znakomici strzelcy \\
    Jazda Kozacka & 50 & 0 & 2 & 0 & 1.0 & Jazda średnia \\
    Czerń & 20 & 0 & 1 & 0 & 1.0 & Liczni, słabo uzbrojeni \\
    Artyleria Kozacka & 5 & 130 & 14 & 25 & 0.25 & Ostrzał obozu \\
    \bottomrule
    \end{tabular}
    }
\end{table}

Wartości parametrów (takie jak \textit{rate\_of\_fire} czy \textit{ammo}) wpływają bezpośrednio na logikę walki. Przykładowo, jednostki o niskiej dyscyplinie (Pospolite Ruszenie - 20, Czerń - 40) są znacznie bardziej podatne na panikę, co zostało opisane w algorytmach w Rozdziale 4.

\subsection{Mapa i teren}
Teren bitwy (plik \texttt{map.tmx}) został przekonwertowany na siatkę kosztów ruchu (\texttt{terrain\_costs}). Koszty te wpływają na prawdopodobieństwo wykonania ruchu przez agenta w danej turze.
\begin{itemize}
    \item Teren podstawowy: koszt 1.0.
    \item Teren trudny (lasy, wzgórza): koszt > 1.0 (zmniejsza szansę na ruch).
    \item Błoto (podczas deszczu): mnożnik kosztu x2.5.
    \item Rzeka: wysoki koszt (ciężko przekraczalne).
    \item Przeszkody (np. ściany obozu): koszt nieskończony (nieprzekraczalne).
\end{itemize}