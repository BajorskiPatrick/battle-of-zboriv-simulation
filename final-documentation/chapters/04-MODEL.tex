\chapter{MODEL FORMALNY} \label{ch:procedure}

\section{Szczegółowy model zachowania agenta}

Logika decyzyjna każdego agenta (\textit{MilitaryAgent}) jest wykonywana w każdej turze symulacji. Proces ten jest deterministyczno-stochastyczny: reguły są stałe, ale ich wynik zależy od rzutów prawdopodobieństwa (np. test morale, szansa na trafienie).

W modelu zaimplementowano zaawansowane mechanizmy psychologii tłumu, symulujące efekt domina na polu bitwy:
\begin{itemize}
    \item \textbf{Panika Łańcuchowa (Chain Panic):} Gdy jednostka nie zda testu morale i rzuci się do ucieczki, wywołuje to falę strachu u pobliskich sojuszników (w promieniu 3 pól), obniżając ich morale. Siła tego efektu jest tym mniejsza, im większa jest dyscyplina obserwatorów (dla zdyscyplinowanych żołnierzy efekt ten jest zerowy).
    \item \textbf{Panika po Śmierci (Death Panic):} Eliminacja jednostki wywołuje silny wstrząs psychiczny u towarzyszy znajdujących się w bezpośrednim sąsiedztwie, co może prowadzić do załamania całej formacji, nawet jeśli nie była ona bezpośrednio atakowana. Siła tego efektu zależy od dyscypliny analogicznie, jak w przypadku Paniki Łańcuchowej.
\end{itemize}

Poniżej przedstawiono kompletny algorytm cyklu życia agenta (Algorytm \ref{alg:main_loop}), procedurę obsługi obrażeń (Algorytm \ref{alg:receive_damage}) oraz dedykowaną procedurę obsługi ucieczki (Algorytm \ref{alg:fleeing}).

\newcommand{\AlgMainLoop}{%
\begin{algorithm}[H]
\caption{\small{\texttt{Główna pętla decyzyjna agenta (Step)}}
\label{alg:main_loop}}
\begin{algorithmic}[1]
\REQUIRE Agent $A$, Grid $G$, Weather $W$
\IF{$A.HP \le 0$} \RETURN \ENDIF
\STATE Zmniejsz $A.fire\_cooldown$ jeśli $> 0$
\STATE $PanicThreshold \gets 25 - (A.Discipline / 5)$
\IF{$A.Morale < PanicThreshold$ \AND $A.State \neq FLEEING$}
	\IF{$Random(0, 100) > A.Discipline$}
		\STATE $A.State \gets FLEEING$
		\STATE \textbf{Call} \textsc{TriggerChainPanic}($A$)
		\STATE \textbf{Call} \textsc{ManageFleeing}($A$)
	\ENDIF
\ENDIF

\IF{$A.State == FLEEING$}
	\IF{brak ścieżki \AND Frakcja == "Armia Koronna"}
		\STATE \textbf{Call} \textsc{ManageFleeing}($A$) \COMMENT{Przelicz cel ucieczki}
	\ENDIF
	\STATE \textbf{Call} \textsc{Move}($A$)
	\RETURN
\ENDIF

\STATE $Enemy \gets$ \textsc{FindEnemy}($A$, radius=$W.visibility$)

\IF{$Enemy$ istnieje}
	\STATE $Dist \gets$ \textsc{Distance}($A$, $Enemy$)
	\IF{$1 < Dist \le A.Range$ \AND $A.RangedDmg > 0$}
		\STATE $Misfire \gets 0.4$ jeśli $W == Rain$ wpp. $0.0$
		\IF{$A.Ammo > 0$}
			\IF{$A.Cooldown \le 0$ \AND $Random() > Misfire$}
				\STATE $A.State \gets ATTACKING$
				\STATE $A.Ammo \gets A.Ammo - 1$
				\STATE \textsc{ReceiveDamage}($Enemy$, $A.RangedDmg$)
			\ENDIF
		\ELSE
			 \STATE \textsc{PathTo}($Enemy$) \COMMENT{Brak amunicji, szarża}
			 \STATE \textsc{Move}($A$)
		\ENDIF
	\ELSIF{$Dist \le 1.5$}
		\STATE $A.State \gets ATTACKING$
		\STATE \textsc{ReceiveDamage}($Enemy$, $A.MeleeDmg$) \COMMENT{Walka wręcz}
	\ELSE
		\STATE \textsc{PathTo}($Enemy$)
		\STATE \textsc{Move}($A$)
	\ENDIF
\ELSE
	\STATE $A.State \gets MOVING\_TO\_STRATEGIC$
	\STATE \textsc{PathTo}($A.StrategicTarget$)
	\STATE \textsc{Move}($A$)
\ENDIF
\end{algorithmic}
\end{algorithm}
}

\newcommand{\AlgReceiveDamage}{%
\begin{algorithm}[H]
\caption{\small{\texttt{Procedura otrzymania obrażeń (Receive Damage)}}
\label{alg:receive_damage}}
\begin{algorithmic}[1]
\REQUIRE Agent $A$, Damage $D$
\STATE $Reduction \gets Random(0, A.Defense / 2)$
\STATE $ActualDmg \gets \max(1, D - Reduction)$
\STATE $A.HP \gets A.HP - ActualDmg$
\STATE $MoraleLoss \gets ActualDmg \times 1.5$
\IF{$A.Discipline > 80$}
	\STATE $MoraleLoss \gets MoraleLoss \times 0.7$
\ENDIF
\STATE $A.Morale \gets A.Morale - MoraleLoss$

\IF{$A.HP \le 0$}
	\STATE \textbf{Call} \textsc{TriggerDeathPanic}($A$)
	\STATE Oznacz agenta jako martwego
\ENDIF
\end{algorithmic}
\end{algorithm}
}

\newcommand{\AlgManageFleeing}{%
\begin{algorithm}[H]
\caption{\small{\texttt{Procedura wyznaczania celu ucieczki (Manage Fleeing)}}
\label{alg:fleeing}}
\begin{algorithmic}[1]
\REQUIRE Agent $A$, Grid $G$
\STATE $CurrentPos \gets A.Position$
\STATE $DistToEdge \gets$ minimum dystansu do krawędzi (L, R, T, B)

\IF{$A.Faction ==$ "Armia Koronna"}
	\STATE $TargetCenter \gets$ Najbliższe \textbf{niezapełnione} Centrum Leczenia
	\STATE $ShouldFleeToEdge \gets True$
    
	\IF{$TargetCenter$ istnieje}
		\STATE $DistToCenter \gets$ \textsc{Distance}($CurrentPos$, $TargetCenter$)
		\IF{$DistToCenter \le 2 \times DistToEdge$}
			\STATE $ShouldFleeToEdge \gets False$
		\ENDIF
	\ENDIF
    
	\IF{NOT $ShouldFleeToEdge$}
		\STATE $Entrance \gets$ \textsc{GetEntrance}($TargetCenter$)
		\IF{$CurrentPos$ na $Entrance$}
			\STATE Wejdź głębiej w strefę leczenia
		\ELSE
			\STATE \textsc{CalculatePath}($Entrance$)
		\ENDIF
	\ELSE
		\STATE \textsc{CalculatePath}(Najbliższa Krawędź Mapy)
	\ENDIF
\ELSE
	\STATE \textsc{CalculatePath}(Najbliższa Krawędź Mapy)
\ENDIF
\end{algorithmic}
\end{algorithm}
}


\AlgMainLoop

\AlgReceiveDamage

\section{Algorytm ucieczki i regeneracji}

Unikalnym elementem modelu jest zróżnicowana strategia ucieczki. Jednostki Kozackie zawsze uciekają do krawędzi mapy. Jednostki Koronne podejmują decyzję: jeśli Obóz (Centrum Leczenia) jest bliżej niż krawędź mapy i posiada wolne miejsca, agent spróbuje się tam schronić, aby zregenerować siły (HP i Morale).

\AlgManageFleeing